\documentclass[10pt]{article}

\usepackage{eucommon}

% Define frame sizes and positioning
% First page
\newlength{\fhDowRestrictions} \setlength\fhDowRestrictions{6\baselineskip}
\newlength{\frameDowRestrictionsY} \setlength\frameDowRestrictionsY{\calc{\textheight - \fhDowRestrictions}}

\newlength{\fhCasusBelli} \setlength\fhCasusBelli{12\baselineskip}
\newlength{\frameCasusBelliY} \setlength\frameCasusBelliY{\calc{\frameDowRestrictionsY - \fhCasusBelli - \frameToFrameSpacing}}

\newlength{\fhHreIntWars} \setlength\fhHreIntWars{4\baselineskip}
\newlength{\frameHreIntWarsY} \setlength\frameHreIntWarsY{\calc{\frameCasusBelliY - \fhHreIntWars - \frameToFrameSpacing}}

\newlength{\fhDefHre} \setlength\fhDefHre{14\baselineskip}
\newlength{\frameDefHreY} \setlength\frameDefHreY{\calc{\frameHreIntWarsY - \fhDefHre - \frameToFrameSpacing}}

\newlength{\fhReceivingCta} \setlength\fhReceivingCta{\calc{\frameDefHreY - \frameToFrameSpacing}}

\newlength{\fhCallToArms} \setlength\fhCallToArms{23\baselineskip}

\newlength{\fhFirstPageLeftCol} \setlength\fhFirstPageLeftCol{\calc{\textheight - \fhCallToArms - \frameToTextSpacing}}
\newlength{\firstPageLeftColY} \setlength\firstPageLeftColY{\calc{\textheight - \fhFirstPageLeftCol}}

%Second page
\newlength{\fhMilitaryAccess} \setlength\fhMilitaryAccess{15\baselineskip}
\newlength{\frameMilitaryAccessY} \setlength\frameMilitaryAccessY{\calc{\textheight - \fhMilitaryAccess}}

\newlength{\fhSecondPageLeftCol} \setlength\fhSecondPageLeftCol{\calc{\textheight - \fhMilitaryAccess - \frameToTextSpacing}}

\newlength{\fhShipsInPort} \setlength\fhShipsInPort{11\baselineskip}
\newlength{\frameShipsInPortY} \setlength\frameShipsInPortY{\calc{\textheight - \fhShipsInPort}}

\newlength{\fhNavalBridge} \setlength\fhNavalBridge{9\baselineskip}

\newlength{\fhSecondPageMiddleCol} \setlength\fhSecondPageMiddleCol{\calc{\textheight - \fhShipsInPort - \fhNavalBridge - 2\frameToTextSpacing}}
\newlength{\secondPageMiddleColY} \setlength\secondPageMiddleColY{\calc{\fhNavalBridge + \frameToTextSpacing}}

\newlength{\fhWarCapacities} \setlength\fhWarCapacities{17\baselineskip}
\newlength{\frameWarCapacitiesY} \setlength\frameWarCapacitiesY{\calc{\textheight - \fhWarCapacities}}

\newlength{\fhSecondPageRightCol} \setlength\fhSecondPageRightCol{\calc{\textheight - \fhWarCapacities - \frameToTextSpacing}}

% Third Page
\newlength{\fhNprWarfare} \setlength\fhNprWarfare{20\baselineskip}
\newlength{\frameNprWarfareY} \setlength\frameNprWarfareY{\calc{\textheight - \fhNprWarfare}}

\newlength{\fhArmiesFleets} \setlength\fhArmiesFleets{8\baselineskip}

\newlength{\fhThirdPageLeftCol} \setlength\fhThirdPageLeftCol{\calc{\textheight - \fhNprWarfare - \fhArmiesFleets - 2\frameToTextSpacing}}
\newlength{\thirdPageLeftColY} \setlength\thirdPageLeftColY{\calc{\fhArmiesFleets + \frameToTextSpacing}}

\newlength{\fhBattleSequence} \setlength\fhBattleSequence{50\baselineskip}

\newlength{\fhBattleTriggers} \setlength\fhBattleTriggers{\calc{\textheight - \fhBattleSequence - \frameToFrameSpacing}}
\newlength{\frameBattleTriggersY} \setlength\frameBattleTriggersY{\calc{\textheight - \fhBattleTriggers}}

% Text Columns
\usepackage{flowfram}
\newflowframe[1]{\columnwidth}{\fhFirstPageLeftCol}{0mm}{\firstPageLeftColY}[firstPageLeftCol]
\newflowframe[2]{\columnwidth}{\fhSecondPageLeftCol}{0mm}{0mm}[secondPageLeftCol]
\newflowframe[2]{\columnwidth}{\fhSecondPageMiddleCol}{\middleColX}{\secondPageMiddleColY}[secondPageMiddleCol]
\newflowframe[2]{\columnwidth}{\fhSecondPageRightCol}{\rightColX}{0mm}[secondPageRightCol]
\newflowframe[3]{\columnwidth}{\fhThirdPageLeftCol}{0mm}{\thirdPageLeftColY}[thirdPageLeftCol]
\newflowframe[3]{\columnwidth}{\textheight}{\middleColX}{0mm}[thirdPageRightCol]

% Text Frames
\newdynamicframe[1]{\columnwidth}{\fhCallToArms}{0mm}{0mm}[frameCallToArms]
\newdynamicframe[1]{\fwTwoCols}{\fhDowRestrictions}{\middleColX}{\frameDowRestrictionsY}[frameDowRestrictions]
\newdynamicframe[1]{\fwTwoCols}{\fhCasusBelli}{\middleColX}{\frameCasusBelliY}[frameCasusBelli]
\newdynamicframe[1]{\fwTwoCols}{\fhHreIntWars}{\middleColX}{\frameHreIntWarsY}[frameHreIntWars]
\newdynamicframe[1]{\fwTwoCols}{\fhDefHre}{\middleColX}{\frameDefHreY}[frameDefHre]
\newdynamicframe[1]{\fwTwoCols}{\fhReceivingCta}{\middleColX}{0mm}[frameReceivingCta]

\newdynamicframe[2]{\columnwidth}{\fhMilitaryAccess}{0mm}{\frameMilitaryAccessY}[frameMilitaryAccess]
\newdynamicframe[2]{\columnwidth}{\fhNavalBridge}{\middleColX}{0mm}[frameNavalBridge]
\newdynamicframe[2]{\columnwidth}{\fhShipsInPort}{\middleColX}{\frameShipsInPortY}[frameShipsInPort]
\newdynamicframe[2]{\columnwidth}{\fhWarCapacities}{\rightColX}{\frameWarCapacitiesY}[frameWarCapacities]

\newdynamicframe[3]{\columnwidth}{\fhNprWarfare}{0mm}{\frameNprWarfareY}[frameNprWarfare]
\newdynamicframe[3]{\columnwidth}{\fhArmiesFleets}{0mm}{0mm}[frameArmiesFleets]
\newdynamicframe[3]{\fwTwoCols}{\fhBattleTriggers}{\middleColX}{\frameBattleTriggersY}[frameBattleTriggers]
\newdynamicframe[3]{\fwTwoCols}{\fhBattleSequence}{\middleColX}{0mm}[frameBattleSequence]

% Arrows
\newdynamicframe[1]{\textwidth}{\textheight}{\pagemarginleft}{-\pagemarginbottom}[frameArrowsPageOne]
\newdynamicframe[2]{\textwidth}{\textheight}{\pagemarginleft}{-\pagemarginbottom}[frameArrowsPageTwo]
\newdynamicframe[3]{\textwidth}{\textheight}{\pagemarginleft}{-\pagemarginbottom}[frameArrowsPageThree]

% Table colors
\usepackage{colortbl}
\usepackage{xcolor}
\ifrenderbw
	\definecolor{tblbgRecruitRegular}{RGB}{255,255,255}
	\definecolor{tblbgRecruitMerc}{RGB}{255,255,255}
	\definecolor{tblbgRecruitAllied}{RGB}{255,255,255}
\else
	\definecolor{tblbgRecruitRegular}{RGB}{217,238,204}
	\definecolor{tblbgRecruitMerc}{RGB}{253,228,151}
	\definecolor{tblbgRecruitAllied}{RGB}{222,221,222}
\fi

\definecolor{colorMilitary}{RGB}{239,140,141}

\begin{document}
\addbackground
\addfooter

\begin{dynamiccontents*}{frameDowRestrictions}\begin{eubox}{\fhDowRestrictions}
	\begin{multicols}{2}
		\subsubsection*{\zsavepos{arrowDowRestrictionsEnd}Restrictions on DoW \normal{(p.~22)}}
		\begin{enumerate}[label=\strong{\alph*}.]
			\begin{multicols}{2}
				\item Your Ally
				\item Truce
			\end{multicols}
			\item PR who has Passed
			\item NPR Ally of PR who matches (\strong{b}) or (\strong{c})
			\item HRE Member at Peace with Emperor if Emperor matches (\strong{a}), (\strong{b}) or (\strong{c})
			\item Distant Realm undiscovered by you
			\item During an Interregnum
		\end{enumerate}
		\smallheading{Exceptions:}
		\begin{itemize}
			\item If you have \disputedsuccession on target and use Disputed Succession CB, then (\strong{a}) and (\strong{g}) do not apply, but Alliance ends
		\end{itemize}
	\end{multicols}
\end{eubox}\end{dynamiccontents*}

\begin{dynamiccontents*}{frameCasusBelli}\begin{eubox}{\fhCasusBelli}
	\begin{multicols}{2}
		\subsubsection*{\zsavepos{arrowCbEnd}Casus Belli \normal{(p.~22)}}
		\strong{Conquest (Claim)} -- Have \claim in Area where target Owns Provinces\\
		\strong{Call to Arms} -- Receive a \emph{CtA}\\
		\strong{General CB} -- Have CB token target\\
		\strong{Event} -- Event that lets you Declare War\\
		\begin{itemize}
			\item Also negates penalty for DoW on \marriage
		\end{itemize}
		\strong{Disputed Succession} -- Any \disputedsuccession on target\\
		\begin{itemize}
			\item Also against PRs at War with the target
			\item Also negates penalty for DoW on \marriage
		\end{itemize}
		\strong{Excommunication} -- You are Catholic and the target is \emph{Excommunicated}\\
		\strong{Holy War (Crusade)}\\
		\begin{itemize}
			\item If you have \idea{Deus Vult} Idea and target
			\begin{itemize}
				\item Is Adjacent to you, \conj{and}
				\item Has different State Religion (except other Christians)
			\end{itemize}
			\item If you are Catholic
			\begin{itemize}
				\item Target Realm is a target of a \emph{Crusade}
				\item Tag \emph{Committed to Crusade} slot when using this CB
			\end{itemize}
		\end{itemize}
		\strong{Imperial Liberation} -- You are the Emperor and the non-HRE member target Controls Provinces or has Vassals in HRE
	\end{multicols}
\end{eubox}\end{dynamiccontents*}

\begin{dynamiccontents*}{frameHreIntWars}\begin{eubox}{\fhHreIntWars}
	\begin{multicols}{2}
		\subsubsection*{\zsavepos{arrowHreIntWarsEnd}HRE Int. Wars with no CB \normal{(p.~45)}}
		\begin{itemize}
			\item Emp.'s DoW on Subject
			\begin{itemize}
				\item Lose 1\authority
				\item Remove 3\influence from HRE Areas
			\end{itemize}
			\item Subject's DoW on another Subject
			\begin{itemize}
				\item Human Emperor must place CB on Aggressor's Capital
				\item \botrule{\hspacezero\zsavepos{arrowBotDefSubjectStart}Bot Emp. defends targeted Subj. (p.~6)}
			\end{itemize}
		\end{itemize}
	\end{multicols}
\end{eubox}\end{dynamiccontents*}

\begin{dynamiccontents*}{frameDefHre}\begin{eubox}{\fhDefHre}
	\begin{multicols}{2}
		\subsubsection*{\zsavepos{arrowHreCtaEnd}Defending the HRE \normal{(p.~44)}}
		\smallheading{External Realm's DoW on Imp. Subject}
		\begin{itemize}
			\item PR Emperor receives \emph{Defensive CtA} if\zsavepos{arrowBotDefSubjectEnd}
			\begin{itemize}
				\item \authority ≥ 1, \conj{and}
				\item They are at Peace with the Subject
			\end{itemize}
			\item If the Emperor accepts
			\begin{itemize}
				\item Apply \dprime Accepting a CtA\dprime procedure\zsavepos{arrowEmpAcceptCtAStart}
				\item Activate \emph{Defending the HRE}
			\end{itemize}
			\item If the Emperor refuses
			\begin{itemize}
				\item Lose 1\authority (no normal penalties)
			\end{itemize}
		\end{itemize}
		\smallheading{External Realm's DoW on the Emperor}
		\begin{itemize}
			\item If Emperor's Capital is in HRE
			\begin{itemize}
				\item May activate \emph{Defending the HRE}
				\begin{itemize}
					\botrule{\item Bot Emperor activates it (p.~4)}
				\end{itemize}
			\end{itemize}
		\end{itemize}
		\subsubsection*{Activating Def. the HRE \normal{(p.~44)}}
		\begin{itemize}
			\item Tag \emph{Defending the HRE} slot
			\item If \strong{human PR is Emperor}, add NPR Units to \strong{Imperial} \manpower = Emperor's \influence (incl. Imperial \influence) in Elec. Areas (max~8)
			{\botrules
			\item If a \strong{Bot is Emperor} (p.~6)
			\begin{itemize}
				\item Gain \botpower = \authority, if activating due to \emph{CtA}
			\end{itemize}
			}
			\item \strong{Human Imperial Subject} must
			\begin{itemize}
				\item Exhaust 2\manpower (max ½ of total \manpower), \conj{or}
				\item Lose 6\ducats (max ½ of Tax Inc.), \conj{or}
				\item Lose \p1, \conj{or}
				\item Place CB on Aggressor's Capital
			\end{itemize}
			\item \botrule{\strong{Bot Imperial Subject} loses 1\botpower, unless at War, including this DoW (p.~6)}
		\end{itemize}
	\end{multicols}
\end{eubox}\end{dynamiccontents*}

\begin{dynamiccontents*}{frameCallToArms}\begin{eubox}{\fhCallToArms}
	\actionHeadingNoSpace{\zsavepos{arrowCtAEnd}Call to Arms \normal{(0-2\influence per \ally) (p.~13)}}
	\begin{itemize}
		\item Call Allies to join your War (Minor Act.)
		\item Only during a DoW
		\begin{itemize}
			\item Your own, \conj{or}
			\item \reaction~-- Targeting you or your NPR Ally
		\end{itemize}
		\item Calling a PR Ally has no cost\zsavepos{arrowReceiveCtAStart}
		\item Only def. may call PR Allied to both sides
		\item To call an NPR, remove \influence from its Areas
		\begin{itemize}
			\item If \emph{Offensive CtA}, 2\influence
			\item If \emph{Defensive CtA}, 1\influence
			\item If \strong{Distant NPR}, may use \colonists instead
		\end{itemize}
		\item NPR Allies can only be called if they are
		\begin{itemize}
			\item At Peace, \conj{and}
			\item Adjacent to you or your new Enemy
		\end{itemize}
		\botrule{\item Bot sends \emph{Def. CtA}s to \allies Adj. to Aggres.}
		\item For each NPR Ally called to arms
		\begin{itemize}
			\item Flip the \alliance to \activeally
			\item Add Allied Units to Available \manpower equal to ½ of Tax Value of the Ally + its Vassals (max 5)
			\item Gain 1\milpower if Ally is Adj. to a new Enemy
		\end{itemize}
		{\botrules
		\begin{itemize}
			\item Bot gains 2\botpower if \emph{Def. CtA} (p.~4)
		\end{itemize}
		}
\end{itemize}
\end{eubox}\end{dynamiccontents*}

\begin{dynamiccontents*}{frameReceivingCta}\begin{eubox}{\fhReceivingCta}
	\begin{multicols}{2}
		\subsubsection*{\zsavepos{arrowReceiveCtAEnd}Receiving a CtA \normal{(p.~32-33)}}
		\begin{itemize}
			\item \emph{Defensive CtA}s can always be accepted
			\item \emph{Offensive CtA}s must be refused in case of DoW restrictions
			{\botrules
			\item Bot accepts \emph{Defensive CtA}s, unless at War with an Opponent (p.~4)
			\item Bot always refuses \emph{Offensive CtA}s (p.~4)
			}
		\end{itemize}
		\smallheading{Accepting a CtA\zsavepos{arrowEmpAcceptCtAEnd}}
		\begin{itemize}
			\item If \emph{Offensive CtA}, place War tokens on your Ally's Enemies
			\item If \emph{Defensive CtA}
			\begin{itemize}
				\item Enemy places War tokens on you
				\item If from NPR, you may
				\begin{itemize}
					\item Make them Active Ally or not
					\item Send \emph{Def. CtA} to other NPR Allies
				\end{itemize}
				\item If you are Allied to a PR on opposing side, this Alliance ends
			\end{itemize}
		\end{itemize}
		\columnbreak

		\smallheading{Refusing a CtA}
		\begin{itemize}
			\item Remove \alliance
			\item If this was an Active Ally
			\begin{itemize}
				\item Lose Allied Units = ½ of Ally's pre-War Tax Value
				\item Enemy must place a War token on your former Ally
			\end{itemize}
			\item If \emph{Defensive CtA}, \conj{and} you have not Passed, \conj{and} you are not already at War
			\begin{itemize}
				\item Lose \p2
				\item Rem. 5\influence from former \ally's Areas
				\item If your former Ally is a PR, they may place a CB on your Capital
			\end{itemize}
			\item Place Truce tokens, unless former Ally is PR who chose to place a CB
		\end{itemize}
		\vfill
	\end{multicols}
\end{eubox}\end{dynamiccontents*}

\subsectionstriped{colorMilitary}{Military Actions}
\actionHeadingNoSpace{Declare War \normal{(1\milpower) (p.~16)}}
\begin{enumerate}
	\item Pick target Realm(s), place War tokens\zsavepos{arrowDowRestrictionsStart}
	\item Lose \stability
	\begin{itemize}
		\item 2\stability per missing CB\zsavepos{arrowCbStart}
		\item 1\stability per your \marriage on targets
	\end{itemize}
	\item \action{Calls to Arms} (in listed order)\zsavepos{arrowCtAStart}
	\begin{enumerate}[label=\alph*.]
		\item You may send \emph{Offensive CtA}s
		\item Target HRE Members might send \emph{Defensive CtA} to the Emperor
		\item Target NPRs send \emph{Defensive CtA}s
		\item Target PRs may send \emph{Defensive CtA}s
	\end{enumerate}
	\item PRs gain 1\milpower if they are
	\begin{itemize}
		\item Target PR, \conj{or}
		\item Accepting \emph{Def. CtA}s from NPRs (unless already at War with Aggressor)
		{\botrules
		\item Bots gain 1\botpower instead (p.~4)
		\begin{itemize}
			\item If then the Bot has < 5/5/6/7 \botpower, it gains \botpower until it reaches 5/5/6/7
			\item If Bot has any Available \manpower, they spend 1\botpower to recruit 7/9/9/11 Units, and check MAC if Army is on the map
		\end{itemize}
		}
	\end{itemize}
	\item Remove all your \influence from target Realms
	\item \hspacezero\zsavepos{arrowBattleFiveStart}Resolve triggered Naval Battles
	\item \hspacezero\zsavepos{arrowBattleSixStart}Resolve triggered Land Battles
	\item If no Battle is triggered, may \action{Activate Units} or \action{Recruit Units} (no \milpower cost)
\end{enumerate}

\begin{dynamiccontents*}{frameArrowsPageOne}
	% Must be before the page ends but after all \savepos in other dynamic frames
	\begin{tikzpicture}
	\useasboundingbox (0,0) rectangle (\paperwidth,\paperheight);
	
	% DoW Restrictions arrow
	\newlength{\dowRestrictionsStartY} \setlength{\dowRestrictionsStartY}{\calc{\zposy{arrowDowRestrictionsStart}sp + \arrowAdjustYText}}
	\newlength{\dowRestrictionsStartX} \setlength{\dowRestrictionsStartX}{\calc{\zposx{arrowDowRestrictionsStart}sp + \arrowToTextSpacing}}
	\newlength{\dowRestrictionsEndY} \setlength{\dowRestrictionsEndY}{\calc{\zposy{arrowDowRestrictionsEnd}sp + \arrowAdjustYBeforeSubsub}}
	\newlength{\dowRestrictionsEndX} \setlength{\dowRestrictionsEndX}{\zposx{arrowDowRestrictionsEnd}sp}
	\newlength{\dowRestrictionsTurnX} \setlength{\dowRestrictionsTurnX}{\arrowXVerticalMidLeft}

	\drawArrow{(\dowRestrictionsStartX, \dowRestrictionsStartY) -- (\dowRestrictionsTurnX,\dowRestrictionsStartY) -- (\dowRestrictionsTurnX,\dowRestrictionsEndY) -- (\dowRestrictionsEndX,\dowRestrictionsEndY)}

	% Casus Belli arrows
	\newlength{\cbStartY} \setlength{\cbStartY}{\calc{\zposy{arrowCbStart}sp + \arrowAdjustYText}}
	\newlength{\cbStartX} \setlength{\cbStartX}{\calc{\zposx{arrowCbStart}sp + \arrowToTextSpacing}}
	\newlength{\cbEndY} \setlength{\cbEndY}{\calc{\zposy{arrowCbEnd}sp + \arrowAdjustYBeforeSubsub}}
	\newlength{\cbEndX} \setlength{\cbEndX}{\zposx{arrowCbEnd}sp}
	\newlength{\cbTurnX} \setlength{\cbTurnX}{\arrowXVerticalMidLeft}
	\newlength{\hreIntWarsEndY} \setlength{\hreIntWarsEndY}{\calc{\zposy{arrowHreIntWarsEnd}sp + \arrowAdjustYBeforeSubsub}}
	\newlength{\hreIntWarsEndX} \setlength{\hreIntWarsEndX}{\zposx{arrowHreIntWarsEnd}sp}
	
	\drawArrow{(\cbStartX, \cbStartY) -- (\cbTurnX,\cbStartY) -- (\cbTurnX,\cbEndY) -- (\cbEndX,\cbEndY)}
	\drawArrow{(\cbStartX, \cbStartY) -- (\cbTurnX,\cbStartY) -- (\cbTurnX,\hreIntWarsEndY) -- (\hreIntWarsEndX,\hreIntWarsEndY)}

	% Call to Arms arrows
	\newlength{\ctaStartY} \setlength{\ctaStartY}{\calc{\zposy{arrowCtAStart}sp + \arrowAdjustYList}}
	\newlength{\ctaStartX} \setlength{\ctaStartX}{\calc{\zposx{arrowCtAStart}sp + \arrowAdjustXBeforeListNum}}
	\newlength{\ctaEndY} \setlength{\ctaEndY}{\calc{\zposy{arrowCtAEnd}sp + \arrowAdjustYBeforeSubsub + 10pt}}
	\newlength{\ctaEndX} \setlength{\ctaEndX}{\zposx{arrowCtAEnd}sp}
	\newlength{\hreCtaEndY} \setlength{\hreCtaEndY}{\calc{\zposy{arrowHreCtaEnd}sp + \arrowAdjustYBeforeSubsub}}
	\newlength{\hreCtaEndX} \setlength{\hreCtaEndX}{\zposx{arrowHreCtaEnd}sp}
	\newlength{\ctaTurnY} \setlength{\ctaTurnY}{\calc{\margin + \fhCallToArms + .5\columnsep}}
	\newlength{\ctaTurnX} \setlength{\ctaTurnX}{\calc{\arrowXVerticalMidLeft - 1mm}}
	\newlength{\ctaTurnXTwo} \setlength{\ctaTurnXTwo}{\arrowXVerticalLeft}

	\drawArrow{(\ctaStartX, \ctaStartY) -- (\ctaTurnX,\ctaStartY) -- (\ctaTurnX,\ctaTurnY) -- (\ctaTurnXTwo,\ctaTurnY) -- (\ctaTurnXTwo,\ctaEndY) -- (\ctaEndX,\ctaEndY)}
	\drawArrow{(\ctaStartX, \ctaStartY) -- (\ctaTurnX,\ctaStartY) -- (\ctaTurnX,\hreCtaEndY) -- (\hreCtaEndX,\hreCtaEndY)}
		
	% Receive Call to Arms line
	\newlength{\receiveCtaStartY} \setlength{\receiveCtaStartY}{\calc{\zposy{arrowReceiveCtAStart}sp + \arrowAdjustYText}}
	\newlength{\receiveCtaStartX} \setlength{\receiveCtaStartX}{\calc{\zposx{arrowReceiveCtAStart}sp + \arrowToTextSpacing}}
	\newlength{\receiveCtaEndY} \setlength{\receiveCtaEndY}{\calc{\zposy{arrowReceiveCtAEnd}sp + \arrowAdjustYBeforeSubsub}}
	\newlength{\receiveCtaEndX} \setlength{\receiveCtaEndX}{\zposx{arrowReceiveCtAEnd}sp}
	\newlength{\receiveCtaTurnX} \setlength{\receiveCtaTurnX}{\arrowXVerticalMidLeft}

	\drawArrow{(\receiveCtaStartX, \receiveCtaStartY) -- (\receiveCtaTurnX,\receiveCtaStartY) -- (\receiveCtaTurnX,\receiveCtaEndY) -- (\receiveCtaEndX,\receiveCtaEndY)}

	% Bot defending HRE Subject arrow
	\newlength{\botDefSubjectStartY} \setlength{\botDefSubjectStartY}{\calc{\zposy{arrowBotDefSubjectStart}sp + \arrowAdjustYList}}
	\newlength{\botDefSubjectStartX} \setlength{\botDefSubjectStartX}{\calc{\zposx{arrowBotDefSubjectStart}sp + \arrowAdjustXBeforeList}}
	\newlength{\botDefSubjectEndY} \setlength{\botDefSubjectEndY}{\calc{\zposy{arrowBotDefSubjectEnd}sp + \arrowAdjustYText}}
	\newlength{\botDefSubjectEndX} \setlength{\botDefSubjectEndX}{\calc{\zposx{arrowBotDefSubjectEnd}sp + \arrowToTextSpacing}}
	\newlength{\botDefSubjectTurnX} \setlength{\botDefSubjectTurnX}{\arrowXVerticalMidRight}

	\drawArrow{(\botDefSubjectStartX, \botDefSubjectStartY) -- (\botDefSubjectTurnX,\botDefSubjectStartY) -- (\botDefSubjectTurnX,\botDefSubjectEndY) -- (\botDefSubjectEndX,\botDefSubjectEndY)}

	% Emperor accepts CtA arrow
	\newlength{\empAcceptCtAStartY} \setlength{\empAcceptCtAStartY}{\calc{\zposy{arrowEmpAcceptCtAStart}sp + \arrowAdjustYText}}
	\newlength{\empAcceptCtAStartX} \setlength{\empAcceptCtAStartX}{\calc{\zposx{arrowEmpAcceptCtAStart}sp + \arrowToTextSpacing}}
	\newlength{\empAcceptCtAEndY} \setlength{\empAcceptCtAEndY}{\calc{\zposy{arrowEmpAcceptCtAEnd}sp + \arrowAdjustYText}}
	\newlength{\empAcceptCtAEndX} \setlength{\empAcceptCtAEndX}{\calc{\zposx{arrowEmpAcceptCtAEnd}sp + \arrowToTextSpacing}}
	\newlength{\empAcceptCtATurnX} \setlength{\empAcceptCtATurnX}{\arrowXVerticalMidRight}

	\drawArrow{(\empAcceptCtAStartX, \empAcceptCtAStartY) -- (\empAcceptCtATurnX,\empAcceptCtAStartY) -- (\empAcceptCtATurnX,\empAcceptCtAEndY) -- (\empAcceptCtAEndX,\empAcceptCtAEndY)}

	% Battle arrow (continues in next page)
	\newlength{\battleFiveStartY} \setlength{\battleFiveStartY}{\calc{\zposy{arrowBattleFiveStart}sp + \arrowAdjustYList}}
	\newlength{\battleFiveStartX} \setlength{\battleFiveStartX}{\calc{\zposx{arrowBattleFiveStart}sp + \arrowAdjustXBeforeListNum}}
	\newlength{\battleSixStartY} \setlength{\battleSixStartY}{\calc{\zposy{arrowBattleSixStart}sp + \arrowAdjustYList}}
	\newlength{\battleSixStartX} \setlength{\battleSixStartX}{\calc{\zposx{arrowBattleSixStart}sp + \arrowAdjustXBeforeListNum}}
	\newlength{\battleTurnYThree} \setlength{\battleTurnYThree}{\calc{\margin + \textheight + .5\columnsep}}
	\newlength{\battleTurnXThree} \setlength{\battleTurnXThree}{\arrowXVerticalLeft}

	\drawLine{(\battleFiveStartX, \battleFiveStartY) -- (\battleTurnXThree,\battleFiveStartY) -- (\battleTurnXThree,\battleTurnYThree) -- (\pagewidth,\battleTurnYThree)}
	\drawLine{(\battleSixStartX, \battleSixStartY) -- (\battleTurnXThree,\battleSixStartY) -- (\battleTurnXThree,\battleTurnYThree) -- (\pagewidth,\battleTurnYThree)}
	
\end{tikzpicture}
\end{dynamiccontents*}

\framebreak

\begin{dynamiccontents*}{frameMilitaryAccess}\begin{eubox}{\fhMilitaryAccess}
	\subsubsection*{\zsavepos{arrowMilAccessEnd}Military Access \normal{(p.~25)}}
	\begin{itemize}
		\item In Areas with 1+ Province whose \emph{de jure} or \emph{de facto} owner is \strong{Friendly} or \strong{Enemy}
		\begin{itemize}
			\item Always available
		\end{itemize}
		\item In \strong{Neutral} Areas
		\begin{itemize}
			\item Not available in Areas with your \claims
			\item You must be at War
			\item Remove 1\influence from the Area or pay 3\ducats
			\item If all Prov. in Area are Owned by PRs, you need permission from one of those PRs
		\end{itemize}
		\item In \strong{HRE} while \emph{Def. HRE} is active (p.~44)
		\begin{itemize}
			\item Free for
			\begin{itemize}
				\item Emperor
				\item Anyone at War with Emperor
			\end{itemize}
		\end{itemize}
	\end{itemize}
\end{eubox}\end{dynamiccontents*}

\begin{dynamiccontents*}{frameShipsInPort}\begin{eubox}{\fhShipsInPort}
	\subsubsection*{Ships in Port \normal{(p.~26)}\zsavepos{arrowPortEnd}}
	\begin{itemize}
		\item Max 2 in a Small Port
		\item Max 4 in a Large Port
		\item Max 6 in a Large Port in a single Fleet
		\item Heavy Ships are repaired at Turn/Round end
		\item If a Port ceases to be Friendly due to an ending \alliance
		\begin{itemize}
			\item Ships must move to Adjacent non-Hostile Sea Zone
			\item If can't move, must be disbanded 
		\end{itemize}
	\end{itemize}
\end{eubox}\end{dynamiccontents*}

\begin{dynamiccontents*}{frameNavalBridge}\begin{eubox}{\fhNavalBridge}
	\subsubsection*{\zsavepos{arrowNavalBridgeOneEnd}Naval Bridge \normal{(p.~26)}\zsavepos{arrowNavalBridgeTwoEnd}}
	\begin{itemize}
		\item Across any number of Sea Zones
		\item A Sea Zone may be crossed by up to 3 Units per 1 Friendly Ship in that Sea Zone
		\item Does not count as a space
		\item May include Ships of PR Allies, unless
		\begin{itemize}
			\item That Sea Zone has Enemy Ships, \conj{or}
			\item Disembarking in a Hostile Area where the Ally has no Enemies
		\end{itemize}
	\end{itemize}
\end{eubox}\end{dynamiccontents*}

\begin{dynamiccontents*}{frameWarCapacities}\begin{eubox}{\fhWarCapacities}
	\subsubsection*{War Capacities \normal{(p.~22-23)}\zsavepos{arrowWarCapacityEnd}}
	\begin{itemize}
		\item A Province may contribute to MC/NC once per Turn (but for both)
	\end{itemize}	
	\smallheading{Military Capacity (MC)}
	\begin{itemize}
		\item MC in Area = Tax Value of Own \towns + \vassals in this Area and Adjacent to this Area
		\item \strong{Blocking MC}
		\begin{itemize}
			\item Occupied Provinces
			\item MC from Adjacent Area blocked by Hostile Units in that Area
			\item MC from Provinces only Adj. by Sea blocked by Hostile Sea Zones
		\end{itemize}
	\end{itemize}
	\smallheading{Naval Capacity (NC)}
	\begin{itemize}
		\item NC in a Sea Zone = \# of Own Ports facing this Sea Zone (Large Ports count as 2)
		\item \strong{Blocking NC}
		\begin{itemize}
			\item Occupied Ports
		\end{itemize}
	\end{itemize}
\end{eubox}\end{dynamiccontents*}

\actionHeading{Activate Units \normal{(1\milpower) (p.~16,25-26)}}
\begin{itemize}
	\item Do \strong{Land Activation}
	\begin{itemize}
		\item \strong{Land Movement}, \conj{or}
		\item \strong{Siege}, \conj{or}
	\end{itemize}
	\item Do \strong{Naval Activation}
	\begin{itemize}
		\item \strong{Naval Movement}, \conj{or}
		\item \strong{Undock}
	\end{itemize}
\end{itemize}

\subsubsection*{Land Movement}
\begin{itemize}
	\item Move an Army or a Unit up to 2 spaces
	\begin{itemize}
		\item \hspacezero\zsavepos{arrowMilAccessStart}Check \strong{Military Access} (p.~25)
		\item \hspacezero\zsavepos{arrowNavalBridgeOneStart}May use \strong{Naval Bridge}
		\item Stop when entering a Distant, Hostile or Neutral Area (p.~25)
		\item Stop when disembarking (p.~26)
	\end{itemize}
	\item On \strong{Distant Cont.} only allowed in (p.~26)
	\begin{itemize}
		\item Friendly Areas
		\item Areas with an Enemy Province
		\item Vacant Terr. with your or Enemy \claim
	\end{itemize}
	\item Crossing a \strong{Mountain Border} to a Hostile or Neutral Area (p.~25)
	\begin{itemize}
		\item Action cost pays for first 3 Units
		\item Pay additional 1\milpower per 3 Units
	\end{itemize}
	\item \strong{Army reorganization} may be done at any point during its movement (p.~25)
	\begin{itemize}
		\item May pick up or drop off Regular Infantry Units
		\item May shift Units between Armies
		\item May be split up or merged with another Army
	\end{itemize}
	\item A \strong{Battle is triggered} when Units enter\zsavepos{arrowBattleOneStart}\\
	an Area containing
	\begin{itemize}
		\item Hostile Units
		\item Enemy NPR Provinces (unless there are already Units Hostile to the NPR)
	\end{itemize}
	\item \strong{Optional rule 2: Available Mercenaries}
	\begin{itemize}
		\item Only if activating an Army for Land Movement in your Own Area
		\item May recruit up to 3 Mercenary Units (normal cost)
		\item They must move with the Army
	\end{itemize}
\end{itemize}

\subsubsection*{Naval Movement}
\begin{itemize}
	\item Select 1 Sea Zone or Friendly Port\zsavepos{arrowPortOneStart}\\
	as destination
	\item Move any number of Ships within range to the destination (Ports have limits)
	\begin{itemize}
		\item Ship/Fleet may move up to 2 spaces
		\item May not pass through Hostile or Distant Sea Zones (p.~25)
	\end{itemize}
	\item On \strong{Distant Continents} (p.~26)
	\begin{itemize}
		\item If you have no \claim, \town or \vassal Adj. to Dist. Sea Zone, you must \action{Explore} to enter it
		\item To move across the Pacific Ocean, spend an additional \monarchpower of any type
	\end{itemize}
	\item \strong{Galleys} are disbanded if the Fleet moves to a Sea Zone without */\textdagger\xspace(p.~24)
	\item \strong{Fleet reorganization} may be done at the start and destination (p.~25)
	\begin{itemize}
		\item May pick up or drop off Light Ships
	\end{itemize}
	\item If destination Sea Zone is not Hostile
	\begin{itemize}
		\item Light Ships may occupy vacant \strong{Trade Protection} slots there (p. 25)
	\end{itemize}
	\item \hspacezero\zsavepos{arrowBattleTwoStart}A \strong{Battle is triggered} when destination
	\begin{itemize}
		\item Contains Enemy Ships, \conj{or}
		\item Faces Enemy NPR Ports (unless there already are Ships Hostile to the NPR)
	\end{itemize}
	\item \hspacezero\zsavepos{arrowBattleThreeStart}May choose to \strong{fight Pirates} in a Trade Node Adjacent to Activated Ships (p.~28)
	\item May use \strong{Naval Bridge} if (p.~26)\zsavepos{arrowNavalBridgeTwoStart}
	\begin{itemize}
		\item Destination Sea Zone is part of it, \conj{and}
		\item Moved Land Units are Adjacent to it
	\end{itemize}
\end{itemize}

\subsubsection*{Undock}
\begin{itemize}
	\item Move any number of your Ships from Ports to Adjacent non-Hostile Sea Zones
	\item \hspacezero\zsavepos{arrowBattleFourStart}May choose to \strong{fight Pirates} in a Trade Node Adjacent to Activated Ships (p.~28)
\end{itemize}

\subsubsection*{Siege \normal{(p.~28)}}
\begin{enumerate}
	\item Pick an Area with 1+ Enemy Controlled Provinces, where you have 1+ Units
	\item Calculate total Siege Strength of Units you will use and pay \milpower cost
	\begin{itemize}
		\item Strength (round down) (p. 24): \\
		\infantry = 1, \cavalry = ½, \artillery = 2 
		\item Pay +1\milpower x \# Sieging Unit after the first
	\end{itemize}
	\item Siege total Tax Val. ≤ Siege Strength
	\begin{itemize}
		\item To Siege an Island Province (blue Port), you need 1+ Ship in a Sea Zone it faces
	\end{itemize}
	\item Resolve effects of \idea{Defensive Mentality}
	\item When successfully Sieging
	\begin{itemize}
		\item \strong{Rebel Occupied Province}
		\begin{itemize}
			\item Remove \rebeltown
			\item Remove \unrest
		\end{itemize}
		\item \strong{NPR Province}
		\begin{itemize}
			\item Add Occupied token
			\item Add your \town (with \unrest)
		\end{itemize}
		\item \strong{Hostile PR's \town/\vassal}
		\begin{itemize}
			\item Add your \town (with \unrest) on top of it
			\item That player must cover a slot on their Town/Vassal track with a \cube
		\end{itemize}
		\item \strong{Enemy Occupied Province} whose Lawful Owner is Friendly or Neutral
		\begin{itemize}
			\item Remove Occupier's \town
		\end{itemize}
		\item \strong{Rebel/Enemy Occupied Province} whose Lawful Owner is your Enemy
		\begin{itemize}
			\item Replace Occupier's \town/\rebeltown with your \town (with \unrest)
		\end{itemize}
	\end{itemize}
	\item Ships move out of successfully Sieged Ports and may trigger a Battle
	\item Players regaining Control of Provinces remove \cubes from Town/Vassal track
\end{enumerate}

\actionHeading{Suppress Unrest \normal{(1\milpower per \unrest) (p.~17)}}
\begin{itemize}
	\item \town/\vassal may not be Occupied
	\item Area may not contain any Hostile Units
\end{itemize}

\begin{dynamiccontents*}{frameArrowsPageTwo}
	\begin{tikzpicture}
	\useasboundingbox (0,0) rectangle (\paperwidth,\paperheight);

	% Military Access arrow
	\newlength{\milAccessStartY} \setlength{\milAccessStartY}{\calc{\zposy{arrowMilAccessStart}sp + \arrowAdjustYList}}
	\newlength{\milAccessStartX} \setlength{\milAccessStartX}{\calc{\zposx{arrowMilAccessStart}sp + \arrowAdjustXBeforeList}}
	\newlength{\milAccessEndY} \setlength{\milAccessEndY}{\calc{\zposy{arrowMilAccessEnd}sp + \arrowAdjustYBeforeSubsub}}
	\newlength{\milAccessEndX} \setlength{\milAccessEndX}{\zposx{arrowMilAccessEnd}sp}
	\newlength{\milAccessTurnX} \setlength{\milAccessTurnX}{\arrowXVerticalLeft}
	
	\drawArrow{(\milAccessStartX, \milAccessStartY) -- (\milAccessTurnX,\milAccessStartY) -- (\milAccessTurnX,\milAccessEndY) -- (\milAccessEndX,\milAccessEndY)}

	% Naval Bridge arrow from Land Movement
	\newlength{\navalBridgeOneStartY} \setlength{\navalBridgeOneStartY}{\calc{\zposy{arrowNavalBridgeOneStart}sp + \arrowAdjustYList}}
	\newlength{\navalBridgeOneStartX} \setlength{\navalBridgeOneStartX}{\calc{\zposx{arrowNavalBridgeOneStart}sp + \arrowAdjustXBeforeList}}
	\newlength{\navalBridgeOneEndY} \setlength{\navalBridgeOneEndY}{\calc{\zposy{arrowNavalBridgeOneEnd}sp + \arrowAdjustYBeforeSubsub}}
	\newlength{\navalBridgeOneEndX} \setlength{\navalBridgeOneEndX}{\zposx{arrowNavalBridgeOneEnd}sp}
	\newlength{\navalBridgeOneTurnY} \setlength{\navalBridgeOneTurnY}{\calc{\margin - .5\columnsep}}
	\newlength{\navalBridgeOneTurnX} \setlength{\navalBridgeOneTurnX}{\arrowXVerticalLeft}
	\newlength{\navalBridgeOneTurnXTwo} \setlength{\navalBridgeOneTurnXTwo}{\arrowXVerticalMidLeft}
	
	\drawArrow{(\navalBridgeOneStartX, \navalBridgeOneStartY) -- (\navalBridgeOneTurnX,\navalBridgeOneStartY) -- (\navalBridgeOneTurnX,\navalBridgeOneTurnY) --
			   (\navalBridgeOneTurnXTwo,\navalBridgeOneTurnY) -- (\navalBridgeOneTurnXTwo,\navalBridgeOneEndY) -- (\navalBridgeOneEndX,\navalBridgeOneEndY)}

	% Naval Bridge arrow from Naval Movement
	\newlength{\navalBridgeTwoStartY} \setlength{\navalBridgeTwoStartY}{\calc{\zposy{arrowNavalBridgeTwoStart}sp + \arrowAdjustYText}}
	\newlength{\navalBridgeTwoStartX} \setlength{\navalBridgeTwoStartX}{\calc{\zposx{arrowNavalBridgeTwoStart}sp + \arrowToTextSpacing}}
	\newlength{\navalBridgeTwoEndY} \setlength{\navalBridgeTwoEndY}{\calc{\zposy{arrowNavalBridgeTwoEnd}sp + \arrowAdjustYAfterSubsub}}
	\newlength{\navalBridgeTwoEndX} \setlength{\navalBridgeTwoEndX}{\calc{\zposx{arrowNavalBridgeTwoEnd}sp + \arrowToTextSpacing}}
	\newlength{\navalBridgeTwoTurnX} \setlength{\navalBridgeTwoTurnX}{\arrowXVerticalMidRight}
	
	\drawArrow{(\navalBridgeTwoStartX, \navalBridgeTwoStartY) -- (\navalBridgeTwoTurnX,\navalBridgeTwoStartY) -- (\navalBridgeTwoTurnX,\navalBridgeTwoEndY) -- (\navalBridgeTwoEndX,\navalBridgeTwoEndY)}

	% Battle arrows (continues from previous and in next pages)
	\newlength{\battleOneStartY} \setlength{\battleOneStartY}{\calc{\zposy{arrowBattleOneStart}sp + \arrowAdjustYList}}
	\newlength{\battleOneStartX} \setlength{\battleOneStartX}{\calc{\zposx{arrowBattleOneStart}sp + \arrowToTextSpacing}}
	\newlength{\battleTwoStartY} \setlength{\battleTwoStartY}{\calc{\zposy{arrowBattleTwoStart}sp + \arrowAdjustYList}}
	\newlength{\battleTwoStartX} \setlength{\battleTwoStartX}{\calc{\zposx{arrowBattleTwoStart}sp + \arrowAdjustXBeforeList}}
	\newlength{\battleThreeStartY} \setlength{\battleThreeStartY}{\calc{\zposy{arrowBattleThreeStart}sp + \arrowAdjustYList}}
	\newlength{\battleThreeStartX} \setlength{\battleThreeStartX}{\calc{\zposx{arrowBattleThreeStart}sp + \arrowAdjustXBeforeList}}
	\newlength{\battleFourStartY} \setlength{\battleFourStartY}{\calc{\zposy{arrowBattleFourStart}sp + \arrowAdjustYList}}
	\newlength{\battleFourStartX} \setlength{\battleFourStartX}{\calc{\zposx{arrowBattleFourStart}sp + \arrowAdjustXBeforeList}}
	\newlength{\battleTurnY} \setlength{\battleTurnY}{\calc{\margin + \textheight + .5\columnsep}}
	\newlength{\battleTurnX} \setlength{\battleTurnX}{\arrowXVerticalMidLeft}
	
	\drawLine{(\battleOneStartX, \battleOneStartY) -- (\battleTurnX,\battleOneStartY) -- (\battleTurnX,\battleTurnY) -- (\pagewidth,\battleTurnY)}
	\drawLine{(\battleTwoStartX, \battleTwoStartY) -- (\battleTurnX,\battleTwoStartY) -- (\battleTurnX,\battleTurnY) -- (\pagewidth,\battleTurnY)}
	\drawLine{(\battleThreeStartX, \battleThreeStartY) -- (\battleTurnX,\battleThreeStartY) -- (\battleTurnX,\battleTurnY) -- (\pagewidth,\battleTurnY)}
	\drawLine{(\battleFourStartX, \battleFourStartY) -- (\battleTurnX,\battleFourStartY) -- (\battleTurnX,\battleTurnY) -- (\pagewidth,\battleTurnY)}
	\drawLine{(0mm,\battleTurnY) -- (\pagewidth,\battleTurnY)}

	% War capacity arrows (continues in next page)
	\newlength{\warCapacityEndY} \setlength{\warCapacityEndY}{\calc{\zposy{arrowWarCapacityEnd}sp + \arrowAdjustYAfterSubsub}}
	\newlength{\warCapacityEndX} \setlength{\warCapacityEndX}{\calc{\zposx{arrowWarCapacityEnd}sp + \arrowToTextSpacing}}
	
	\drawArrow{(\pagewidth, \warCapacityEndY) -- (\warCapacityEndX,\warCapacityEndY)}
	
	% Ships in Port arrows (continues in next page)
	\newlength{\portOneStartY} \setlength{\portOneStartY}{\calc{\zposy{arrowPortOneStart}sp + \arrowAdjustYText}}
	\newlength{\portOneStartX} \setlength{\portOneStartX}{\calc{\zposx{arrowPortOneStart}sp + \arrowToTextSpacing}}
	\newlength{\portEndY} \setlength{\portEndY}{\calc{\zposy{arrowPortEnd}sp + \arrowAdjustYAfterSubsub}}
	\newlength{\portEndX} \setlength{\portEndX}{\calc{\zposx{arrowPortEnd}sp + \arrowToTextSpacing}}
	\newlength{\portTurnY} \setlength{\portTurnY}{\calc{\margin + \frameWarCapacitiesY - .5\columnsep}}
	\newlength{\portTurnX} \setlength{\portTurnX}{\arrowXVerticalMidRight}
	\newlength{\portTurnYTwo} \setlength{\portTurnYTwo}{\calc{\zposy{arrowPortTwoStart}sp + \arrowAdjustYList}}
	\newlength{\portTurnXTwo} \setlength{\portTurnXTwo}{\arrowXVerticalRight}
	
	\drawArrow{(\portOneStartX, \portOneStartY) -- (\portTurnX,\portOneStartY) -- (\portTurnX,\portEndY) -- (\portEndX,\portEndY)}
	\drawArrow{(\pagewidth,\portTurnYTwo) -- (\portTurnXTwo,\portTurnYTwo) -- (\portTurnXTwo,\portTurnY) -- (\portTurnX,\portTurnY) -- (\portTurnX,\portEndY) -- (\portEndX, \portEndY)}

	\end{tikzpicture}
\end{dynamiccontents*}

\framebreak

\actionHeading{Recruit Units \normal{(1\milpower + X\ducats) (p.~17)}}
\begin{itemize}
	\item May recruit as many as you can afford
	\item May recruit in multiple Areas/Sea Zones
	\item Only Regular Infantry/Light Ships can be deployed outside Armies/Fleets
	\item \strong{Artillery} Units require \idea{Cannons} Idea 
\end{itemize}
\smallheading{\zsavepos{arrowWarCapacityOneStart}Regular Units}
\begin{itemize}
	\item In your or \vassal Areas (up to your MC)
\end{itemize}
\smallheading{\zsavepos{arrowWarCapacityTwoStart}Allied Units}
\begin{itemize}
	\item In your Areas (up to your MC)
	\item In Areas of \activeallies (up to their MC)
\end{itemize}
\smallheading{Mercenary Units \normal{(Max 3 per Turn)}}
\begin{itemize}
	\item In your or \vassal Areas (MC irrelevant)
\end{itemize}
\smallheading{Ships}
\begin{itemize}
	\item 1 Ship per Own Port (2 if Large) (p.~4)
	\item \hspacezero\zsavepos{arrowPortTwoStart}Place in Own Port or Adj. non-Hostile Sea Zone, optionally on vacant Tr. Prot. slots
\end{itemize}
\smallheading{Costs}
\begin{tabularx}{\columnwidth}{ | l | C | C | C | }
	\hline
	\null & \cellcolor{tblbgRecruitRegular} \strong{Regular} & \cellcolor{tblbgRecruitMerc} \strong{Merc.} & \cellcolor{tblbgRecruitAllied} \strong{Allied} \\ \hline
	Infantry & \cellcolor{tblbgRecruitRegular} 2\ducats & \cellcolor{tblbgRecruitMerc} 4\ducats & \cellcolor{tblbgRecruitAllied} free \\ \hline
	Cavalry & \cellcolor{tblbgRecruitRegular} 5\ducats & \cellcolor{tblbgRecruitMerc} 7\ducats & \cellcolor{tblbgRecruitAllied} 3\ducats \\ \hline
	Artillery & \cellcolor{tblbgRecruitRegular} 6\ducats & \cellcolor{tblbgRecruitMerc} 8\ducats & - \\ \hline
	Light Ship & 4\ducats & - & - \\ \hline
	Heavy Ship & 10\ducats & - & - \\ \hline
	Galley & 2\ducats & - & - \\ \hline
\end{tabularx}

\begin{dynamiccontents*}{frameNprWarfare}\begin{eubox}{\fhNprWarfare}
	\subsubsection*{Warfare vs NPRs \normal{(p.~36)}\zsavepos{arrowNPRWarEnd}}
	\begin{itemize}
		\item \hspacezero\zsavepos{arrowWarCapacityThreeStart}\# of def. \strong{NPR Units} = MC or NC
		\begin{itemize}
			\item Land Units are Infantry
			\item Ships are Light Ships
			\item If HRE Area with \strong{NPR Emp.} (p.~45)
			\begin{itemize}
				\item +3× \authority
				\item -2× \# of HRE Areas with non-HRE Units prior to this Turn
			\end{itemize}
		\end{itemize}
		\item \strong{Active Ally} defends with ½ of MC
		\item NPR Provinces on \strong{Distant Continents}
		\begin{itemize}
			\item Double MC/NC for defense
			\begin{itemize}
				\item Except from Areas with \plague
			\end{itemize}
			\item Some Ports are Inactive (grayed out) until they have a \dnpr, \town or \vassal
		\end{itemize}
		\item NPRs defend at normal strength even if not enough tokens in Supply
		\item If \strong{multiple Battles}, assign MC in order:
		\begin{enumerate}
			\item Capital Area and Adj. Sea Zones
			\item Largest Enemy force
			\item First Battle
		\end{enumerate}
	\end{itemize}
\end{eubox}\end{dynamiccontents*}

\begin{dynamiccontents*}{frameArmiesFleets}\begin{eubox}{\fhArmiesFleets}
	\subsubsection*{Armies/Fleets \normal{(p.~24)}}
	\begin{itemize}
		\item To deploy an Army, assign Unit(s) to it 
		\begin{itemize}
			\item From its Area (\action{Land Activ.}), \conj{or}
			\item From Available \manpower (during \action{Recruit})
		\end{itemize}
		\item To deploy a Fleet, assign Ship(s) to it
		\begin{itemize}
			\item From Sea Zone (\action{Naval Activ.}), \conj{or}
			\item From your Supply (during \action{Recruit})
		\end{itemize}
		\item If it becomes empty, remove from map
	\end{itemize}
\end{eubox}\end{dynamiccontents*}

\begin{dynamiccontents*}{frameBattleTriggers}\begin{eubox}{\fhBattleTriggers}
	\subsubsection*{Battle Triggers \normal{(p.~27-28)}\zsavepos{arrowBattleTriggerEnd}}
	\begin{multicols}{2}
		\begin{itemize}
			\item It is specified in \action{Activate Units}, \conj{or}
			\item Units/Ships Hostile to each other end up in the same Area/Sea Zone, \conj{or}
			\item \hspacezero\zsavepos{arrowNPRWarStart}Units are Recruited in an Area containing NPR Prov. which are or become Hostile
			\begin{itemize}
				\item Unless Units Hostile to that NPR were in that Area prior to the current Turn
			\end{itemize}
			\item Order if multiple Battles (p.~22):
			\begin{enumerate}
				\item Naval before Land Battles
				\item Active Player decides
			\end{enumerate}
		\end{itemize}
	\end{multicols}
\end{eubox}\end{dynamiccontents*}

\begin{dynamiccontents*}{frameBattleSequence}\begin{eubox}{\fhBattleSequence}
	\begin{multicols}{2}
		\subsubsection*{Battle Sequence \normal{(p.~26-28)}\zsavepos{arrowBattleSequenceEnd}}
		\begin{itemize}
			\item Ships vacate Trade Prot. slots (p.~28)
			\botrule{\item Bots use Land/Naval Res. charts (p.~16)}
		\end{itemize}
		\smallheading{1. Battle Preparations}
		\begin{itemize}
			\item Emperor may use Imperial \manpower (p.~44)
			\begin{itemize}
				\item Only usable in
				\begin{itemize}
					\item HRE Areas
					\item Emp.'s Areas Adj. by Land to HRE
				\end{itemize}
				\item May not be used when Enemy force consists of only NPR HRE Members
				\item Add as Ally Infantry when Battle starts
			\end{itemize}
			\item Multiple Defenders defend together
			\item If 2+ PR Def., pick \strong{Main Defender}
			\begin{itemize}
				\item Priority for Main Defender selection:
				\begin{enumerate}
					\item \botrule{Humans before Bots (p.~5)}
					\item PR with the most Units
					\item PR who last took a Turn decides
				\end{enumerate}
				\item Only the Main Defender may
				\begin{itemize}
					\item Assign a General to the Battle
					\item Play \emph{Battle Actions}
					\item Roll Dice
				\end{itemize}
				{\botrules
				\item If one of the Defenders is a Bot (p.~6)
				\begin{itemize}
					\item Main Defender gets +3 NPR Ships on their side in Naval Battle
				\end{itemize}
				}
			\end{itemize}
			\item Attacker may \action{Appoint Leader}
			\item Def. may \action{App. General} if in their Realm
			\item May not \action{App. Leader} later in the Battle
			\item Max 1 Leader on each side (p.~25,~27)
			\item If more than 1 Leader, then player may choose which one to use (p.~25)
			\item If only \strong{NPR/Rebel Defenders} with total of 3+ Units (p.~36,~37)
			\begin{itemize}
				\item Draw \milcard
				\item Use as Defender's Leader, if any
			\end{itemize}
			\item Apply Military Ideas effects
		\end{itemize}
		\smallheading{2. Play Battle Actions (\battleaction)}
		\begin{itemize}
			\item Attacker plays all \battleactions before Defender
			\item In each Battle Round, each side may only benefit from 1 use of the same \battleaction (p.~19)
			\item Effects of a \battleaction last for the duration of Battle, unless stated otherwise (p.~26)
			\item \strong{Opt. Rule 4: Helping Hand} (p.~36)
			\begin{itemize}
				\item All PRs may play \battleactions to back NPRs (start from Active PR)
			\end{itemize}
		\end{itemize}
		\smallheading{3. Roll Battle Dice}
		\begin{itemize}
			\item If \strong{Land Battle}
			\begin{itemize}
				\item Default 3\infantry Dice
				\begin{itemize}
					\item 3\infantry/3\cavalry for Muslim PRs (p.~38)
				\end{itemize}
				\item 1 hit per matching unit
				\item By default, \tercios=\infantry
			\end{itemize}
			\item If \strong{Naval Battle}
			\begin{itemize}
				\item Default 3\artillery Dice
				\item Hits = \artillery + \# Heavy Ships
			\end{itemize}
			\item Additional Dice from Leaders and \battleactions
		\end{itemize}
		\smallheading{4. Assign Casualties}
		\begin{itemize}
			\item If \strong{multiple Defenders}, then
			\begin{itemize}
				\item Alternate, largest to smallest faction
				\item Attacker decides ties
			\end{itemize}
			\item If \strong{Land Battle}
			\begin{itemize}
				\item Alternate between Merc., Regular and Allied Units in that order
				\begin{itemize}
					\item PR taking hits chooses within these
				\end{itemize}
				\item Regular Units go to Exhausted \manpower
				\item Discard Mercenaries, Allied Units
			\end{itemize}
			\item If \strong{Naval Battle}
			\begin{itemize}
				\item PR taking hits chooses Ships taking hits
				\item Heavy Ships can take 2 hits
				\begin{itemize}
					\item Lay it on its side after first hit
				\end{itemize}
			\end{itemize}
		\end{itemize}
		\smallheading{5A. Wounded Generals/Admirals}
		\begin{itemize}
			\item If you inflicted 1+ Casualty
			\begin{itemize}
				\item Enemy Leader gets 1\illhealth per your 2\tercios
			\end{itemize}
			\item A Leader receiving the second \illhealth dies
		\end{itemize}
		\smallheading{5B. Captured Enemy Ships}
		\begin{itemize}
			\item Only if you have
			\begin{itemize}
				\item Ships remaining, \conj{and}
				\item Eliminated all Enemy Ships
			\end{itemize}
			\item Capt. 1 Enemy Casualty per \tercios (last roll)
			\begin{itemize}
				\item Enemy decides which Ships 
				\item You may deploy Fleet if available
			\end{itemize}
			\item Capt. Heavy Ships are damaged (p.~24)
		\end{itemize}
		\smallheading{6. Retreat}
		\begin{itemize}
			\item Attacker chooses first, then defender
			\item \strong{NPRs retreat} if outnum., unless (p.~36)
			\begin{itemize}
				\item Fighting alongside Rebels, \conj{or}
				\item In their Capital Area, \conj{or}
				\item In Sea Zone Adj. to Capital Area, \conj{or}
				\item In last Area where they Control Prov.
			\end{itemize}
			\item \strong{Rebels} never retreat (p.~37)
			\item If nobody retreats, then go back to step 2
			\item If PR chooses to Retreat, +1 Casualty
			\item \strong{Retreat destination}
			\begin{itemize}
				\item Attacker -- Previous space(s)
				\item Def. -- Adj. sp. with no Enemy Units
				\begin{itemize}
					\item Military Access rules apply
					\item Each PR may choose diff. dest.
				\end{itemize}
			\end{itemize}
			\item Remove retreating/defending NPR units
		\end{itemize}
		\smallheading{7. Proclaim a Winner}
		\begin{itemize}
			\item The side with Units left in the Area wins
			\item Winner Active PR gains 1\milpower (max 1/Turn)
			\item Return surviving Imperial \manpower (p.~44)
		\end{itemize}
	\end{multicols}
\end{eubox}\end{dynamiccontents*}

\begin{dynamiccontents*}{frameArrowsPageThree}
	\begin{tikzpicture}
	\useasboundingbox (0,0) rectangle (\paperwidth,\paperheight);

	% Battle arrows (continues from previous page)
	\newlength{\battleTriggerEndY} \setlength{\battleTriggerEndY}{\calc{\zposy{arrowBattleTriggerEnd}sp + \arrowAdjustYAfterSubsub}}
	\newlength{\battleTriggerEndX} \setlength{\battleTriggerEndX}{\calc{\zposx{arrowBattleTriggerEnd}sp + \arrowToTextSpacing}}
	\newlength{\battleSequenceEndY} \setlength{\battleSequenceEndY}{\calc{\zposy{arrowBattleSequenceEnd}sp + \arrowAdjustYAfterSubsub}}
	\newlength{\battleSequenceEndX} \setlength{\battleSequenceEndX}{\calc{\zposx{arrowBattleSequenceEnd}sp + \arrowToTextSpacing}}
	\newlength{\battleTurnYContinue} \setlength{\battleTurnYContinue}{\calc{\margin + \textheight + .5\columnsep}}
	\newlength{\battleTurnXTwo} \setlength{\battleTurnXTwo}{\arrowXVerticalMidRight}
	
	\drawArrow{(0mm, \battleTurnYContinue) -- (\battleTurnXTwo,\battleTurnYContinue) -- (\battleTurnXTwo,\battleTriggerEndY) -- (\battleTriggerEndX,\battleTriggerEndY)}
	\drawArrow{(0mm, \battleTurnYContinue) -- (\battleTurnXTwo,\battleTurnYContinue) -- (\battleTurnXTwo,\battleSequenceEndY) -- (\battleSequenceEndX,\battleSequenceEndY)}

	% War capacity arrows (continues from previous page)
	\newlength{\warCapacityOneStartY} \setlength{\warCapacityOneStartY}{\calc{\zposy{arrowWarCapacityOneStart}sp + \arrowAdjustYText}}
	\newlength{\warCapacityOneStartX} \setlength{\warCapacityOneStartX}{\zposx{arrowWarCapacityOneStart}sp}
	\newlength{\warCapacityTwoStartY} \setlength{\warCapacityTwoStartY}{\calc{\zposy{arrowWarCapacityTwoStart}sp + \arrowAdjustYText}}
	\newlength{\warCapacityTwoStartX} \setlength{\warCapacityTwoStartX}{\zposx{arrowWarCapacityTwoStart}sp}
	\newlength{\warCapacityThreeStartY} \setlength{\warCapacityThreeStartY}{\calc{\zposy{arrowWarCapacityThreeStart}sp + \arrowAdjustYList}}
	\newlength{\warCapacityThreeStartX} \setlength{\warCapacityThreeStartX}{\calc{\zposx{arrowWarCapacityThreeStart}sp + \arrowAdjustXBeforeList}}
	\newlength{\warCapacityTurnY} \setlength{\warCapacityTurnY}{\calc{\zposy{arrowWarCapacityEnd}sp + \arrowAdjustYAfterSubsub}}
	\newlength{\warCapacityTurnX} \setlength{\warCapacityTurnX}{\arrowXVerticalLeft}
	
	\drawLine{(\warCapacityOneStartX, \warCapacityOneStartY) -- (\warCapacityTurnX,\warCapacityOneStartY) -- (\warCapacityTurnX,\warCapacityTurnY) -- (0mm,\warCapacityTurnY)}
	\drawLine{(\warCapacityTwoStartX, \warCapacityTwoStartY) -- (\warCapacityTurnX,\warCapacityTwoStartY) -- (\warCapacityTurnX,\warCapacityTurnY) -- (0mm,\warCapacityTurnY)}
	\drawLine{(\warCapacityThreeStartX, \warCapacityThreeStartY) -- (\warCapacityTurnX,\warCapacityThreeStartY) -- (\warCapacityTurnX,\warCapacityTurnY) -- (0mm,\warCapacityTurnY)}
	
	% Ships in Port arrows (continues from previous page)
	\newlength{\portTwoStartY} \setlength{\portTwoStartY}{\calc{\zposy{arrowPortTwoStart}sp + \arrowAdjustYList}}
	\newlength{\portTwoStartX} \setlength{\portTwoStartX}{\calc{\zposx{arrowPortTwoStart}sp + \arrowAdjustXBeforeList}}
	
	\drawLine{(\portTwoStartX, \portTwoStartY) -- (0mm,\portTwoStartY)}
	
	% NPR Warfare arrow
	\newlength{\nprWarStartY} \setlength{\nprWarStartY}{\calc{\zposy{arrowNPRWarStart}sp + \arrowAdjustYList}}
	\newlength{\nprWarStartX} \setlength{\nprWarStartX}{\calc{\zposx{arrowNPRWarStart}sp + \arrowAdjustXBeforeList}}
	\newlength{\nprWarEndY} \setlength{\nprWarEndY}{\calc{\zposy{arrowNPRWarEnd}sp + \arrowAdjustYAfterSubsub}}
	\newlength{\nprWarEndX} \setlength{\nprWarEndX}{\calc{\zposx{arrowNPRWarEnd}sp + \arrowToTextSpacing}}
	\newlength{\nprWarTurnX} \setlength{\nprWarTurnX}{\arrowXVerticalMidLeft}
	
	\drawArrow{(\nprWarStartX, \nprWarStartY) -- (\nprWarTurnX,\nprWarStartY) -- (\nprWarTurnX,\nprWarEndY) -- (\nprWarEndX,\nprWarEndY)}
	
	\end{tikzpicture}
\end{dynamiccontents*}

\end{document}
